\section{Related Work}
\label{sec:related}

While there have been a number of successful recent approaches towards automatic colorization 
\cite{zhang2016colorful,iizuka2016let}, most are focused on the task of converting grayscale images to color. Quite a few 
approaches use a physics-based technique to directly model light refraction~\cite{jordt2014underwater}. Specifically for
restoring color in underwater images, the work of~\cite{torres2005color} uses an energy minimization formulation using a Markov 
Random Field. Most similar to the work proposed in this paper is the recently proposed WaterGAN~\cite{li2017watergan}, 
which uses an adversarial approach towards generating realistic underwater images. Their generator model can be broken down into 
three stages: 1) Attenuation, which accounts for range-dependent attenuation of light. 2) Scattering, which models the haze 
effect caused by photons scattering back towards the image sensor and 3) Vignetting, which produces a shading effect on the 
image corners that can be caused by certain camera lenses. Differentiating from our work, they use a GAN for generating the 
underwater images and use strictly Euclidean loss for color correction, whereas we use a GAN for both. Furthermore, they require 
depth information during the training of WaterGAN, which can be often difficult to attain particularly for underwater 
autonomous robotic applications. Our work only requires images of objects in two separate domains (\emph{e.g.}, underwater 
and terrestrial) throughout the entire process.

Recent work in generative models, specifically GANs, have shown great success in areas such as inpainting 
\cite{pathak2016context}, style transfer \cite{Gatys_2016_CVPR}, and image-to-image translation 
\cite{isola2016image,zhu2017unpaired}. This is primarily due to their ability to provide a more meaningful loss than simply the 
Euclidean distance, which has been shown to produce blurry results. In our work, we structure the problem of estimating 
the true appearance of underwater imagery as a paired image-to-image translation problem, using Generative Adversarial Networks 
(GANs) as our generative model\starnote{this needs a little bit more expansion}. Much like the work of \cite{isola2016image}, we 
use image pairs from two domains as input and ground truth. \starnote{Somewhere I think we should mention that unpaired image to 
image translation is more difficult, and that cyclegan is a good way to generate a dataset in such a way to do this with image 
pairs.}